% Article template for Mathematics Magazine
% Revised 7/2002  Thanks for Greg St. George
\documentclass[12pt]{article}
\usepackage{amssymb}
\usepackage[ngerman]{babel}
\usepackage[utf8]{inputenc}
\usepackage{amsmath}
\usepackage{amsthm}
\usepackage{graphicx}
\usepackage{tikz}
\usepackage{dot2texi}
\usetikzlibrary{shapes,arrows}
\renewcommand{\baselinestretch}{1.2}
%This is the command that spaces the manuscript for easy reading


%todo
\usepackage[colorinlistoftodos,prependcaption,textsize=tiny]{todonotes}
\usepackage{xargs}                      % Use more than one optional parameter in a new commands
\newcommandx{\QUESTION}[2][1=]{\todo[linecolor=none,backgroundcolor=blue!15,bordercolor=none,#1]{\textbf{QUESTION: }#2}}


\newtheoremstyle{break}
  {\topsep} {\topsep}%
  {}{}%
  {\bfseries}{:}%
  {\newline}{}%
\theoremstyle{break}
\newtheorem{zeige}{Zeige}
\newtheorem{definition}{Definition}
\newtheorem{bsp}{Beispiel}
\newtheorem{thm}{Theorem}



\begin{document}
%\thispagestyle{empty}
\title{AOTidf Bagger Game}
\author{Denis Erfurt, Tobias Behrens}
\maketitle

\section{Voraussetzungen}

\begin{definition}[CSGS]
  Eine Coalitional Skill Game Setting (CSGS) Signatur\\
  $\sigma_{CSGS}:=\sigma_{Ar}\cup\{Agent_{/1}, Baustelle_{/1}, supply_{/2}, demand_{/2}, budget_{/1}, kosten_{/3} \}$. Dabei steht $\sigma_{Ar}$ für die Signatur der Standardarithmetik mit $0,1,+,*\in\sigma_{Ar}$ \\
  Im weiteren werden Funktionen mit kleinem Anfangsbuchstaben, sowie Relationen mit einem großen Anfangsbuchstaben geschrieben. Ebenfalls sind alle funktionen total und mit 0 initialisiert, falls für eine Struktur und eingabeparameter nicht näher definiert sind.
  Ebenfalls werden wir die Prädikatenschreibweise und die Mengenschreibweise equivalent verwenden: $a\in A \equiv A(a)$
    \\ \textbf{Intuition} \\
    $Agent(x) :\Leftrightarrow\text{ x ist ein Agent}$ \\
    $Baustelle(x) :\Leftrightarrow\text{ x ist eine Baustelle}$ \\
    $supply(x, t)\mapsto n :\Leftrightarrow\text{ Agent x besitzt n Einheiten vom typ t}$ \\
    $demand(x, t)\mapsto n :\Leftrightarrow\text{ Baustelle x benötigt n Einheiten vom typ t}$ \\
    $budget(x)\mapsto n :\Leftrightarrow$
    Baustelle x ist maximal bereit einen Gewin von n bei fertigstellung auszuzahlen \\
    $kosten(t, n, x, y):\Leftrightarrow$ Funktion die die Kosten für einen Agent x für den Transport n Ressourcen t an die Baustelle y berrechnet.
\end{definition}

\begin{bsp}
  Sei $\mathcal{A}$ eine $\sigma_{CSGS}-$Struktur:\\
  $Agent^\mathcal{A} := \{a_1, a_2, a_3\}$ \\
  $Baustelle^\mathcal{A} := \{b_1, b_2\}$ \\
  $ \\
  supply^\mathcal{A}(a_1, t_1)\mapsto 2 \\
  supply^\mathcal{A}(a_1, t_2)\mapsto 7 \\
  supply^\mathcal{A}(a_2, t_1)\mapsto 3 \\
  supply^\mathcal{A}(a_2, t_2)\mapsto 5 \\
  supply^\mathcal{A}(a_3, t_1)\mapsto 20 \\
  supply^\mathcal{A}(a_3, t_2)\mapsto 5
  $ \\ \\
  $
  demand^\mathcal{A}(b_1, t_1)\mapsto 10 \\
  demand^\mathcal{A}(b_1, t_2)\mapsto 5\\
  demand^\mathcal{A}(b_2, t_1)\mapsto 2\\
  demand^\mathcal{A}(b_2, t_2)\mapsto 2
  $ \\ \\
  $
  budget^\mathcal{A}(b_1)\mapsto 10 \\
  budget^\mathcal{A}(b_2)\mapsto 3
  $ \\
\end{bsp}

\begin{definition}[CSGS]
  Eine Coalitioal Skill Game Signatur \\$\sigma_{CSG}:=\sigma_{CSGS}\cup\{M_{/4}, v_{/3}\}$.
    \\ \textbf{Intuition} \\
    $m(x, t, y)\mapsto n:\Leftrightarrow$ Agent x sendet n Ressourcen des Types t an die Baustelle y \\
    $v(x,y)\mapsto n:\Leftrightarrow$ Agent x bekommt von Baustelle y eine vergütung von n
\end{definition}


\begin{bsp}
  Sei $\mathcal{A}'$ eine erweiterung der $\mathcal{A}$ Struktur zu einer $\sigma_{CSG}-$Struktur mit:\\
  $
  m(a_1, t_1, b_1) \mapsto 2 \\
  m(a_1, t_2, b_1) \mapsto 5 \\
  m(a_2, t_1, b_1) \mapsto 3 \\
  m(a_3, t_1, b_1) \mapsto 5 \\
  m(a_3, t_1, b_2) \mapsto 2 \\
  m(a_3, t_2, b_2) \mapsto 2
  $ \\
  Sonst $m(x,t,y)\mapsto 0$
\end{bsp}


\section{Problemstellung}

Gesucht wird ein Mechanismus der bei eingabe einer $\sigma_{CSGS}$-Struktur eine $\sigma_{CSG}$-Struktur berechnet die bestimmte noch zu definierende Eigenschaften erfüllt.

\section{Ergebnisse}

\begin{definition}
  Sei $K\subseteq A$ eine beliebige Koalition. Wir definieren die Menge $PM(K)$ der potentiellen Matches:
  \begin{eqnarray}
    PM(K) := \{ &\\
    &m: Agent\times Resource \times Baustelle \rightarrow \mathbb{N}\ |\\
    & \phi_{Match}(m) \land \\ & \forall x.\forall t.\forall y. m(x,t, y) > 0 \rightarrow x\in K\} \label{matchown}
  \end{eqnarray}
  Dabei definiert $\varphi_{Match}(m)$ die validen Matches:
  \begin{eqnarray}
    \varphi_{Match}(m) := \forall x.\forall t.(\sum_{y\in Baustelle} m(x,t,y))\leq supply(x,t)
  \end{eqnarray}
  Die Zeile (\ref{matchown}) sagt das nur solche Matches betrachtet werden, deren ''aktive'' Agenten auch teil der Koalition sind.
\end{definition}

\begin{definition}
  Sei K eine beliebige Koalition, dann definiert $B(K)$ die Menge der möglichen Baustellen, die von K gleichzeitig gebaut werden können.
  \begin{eqnarray}
    B(K) := \{ B_k\subseteq Baustelle\ |\\
    \exists m\in PM(K).\forall b\in B_k.\forall t.(\sum_{x\in K}m(x,t,b))\geq demand(b, t)\}
  \end{eqnarray}
\end{definition}

Da es sich um ein ''einfaches'' Spiel handelt, mach es sinn einen Agenten nur einer Koalition zuzuordnen:
\begin{equation}
  K\neq S \Rightarrow K\cap S =\emptyset \label{koalitiondisjunct}
\end{equation}

Hierraus folgt notwendigerweise, dass eine Baustelle in einem Spiel nur von einer Koalition gebaut werdenn kann, da diese pro Spiel nur einmal gebaut werden kann. Um dieses formal festzuhalten defenieren wir eine Outcome Relation:

\begin{definition}
  Seien K, S Koalitionen. Eine Outcome Menge weist beiden Koalitionen ein Tupel der Baustellen zu, die sie jeweils zur gleichen Zeit bauen können:
\begin{eqnarray}
  Outcome(K,S) := \{(B_K,B_S)\subseteq Baustelle\times Baustelle\ |\\B_K\in B(K) \land \\ B_S\in B(S)\land\\B_K\cap B_S =\emptyset \}
\end{eqnarray}
\end{definition}

Zur weiteren vereinfachung definieren wir uns noch eine Bewertungsfunktion für eine Menge von Baustellen:


\begin{definition}
  Sei $B\subseteq Baustelle$ beliebig. Dann definiert sich der wert von B:
  \begin{equation}
    v(B):=\sum_{b\in B} budget(b)
  \end{equation}
\end{definition}


\begin{definition}
  Nun können wir den Wert betrachten, die zwei Koalitionen erzielen können. Seien K,S beliebige Koalition:
\begin{equation}
  v(K) + v(S) := \max_{(B_K, B_S)\in Outcome(K,S)}(v(B_K)+v(B_S))
\end{equation}
Dadurch dass die Baustellen die von einer Koalition gebaut werden nicht mehr von einer anderen koalition gebaut werden können, müssen wir diese zeitgleich betrachten.\\
\textbf{Bemerkung}: Definitionsgemäs ist das ergäbniss pareto effizient.
\end{definition}

Exestiert eine Koalition isoliert, lässt sich diese auch isoliert betrachten:
\begin{definition}
  Sei K eine beliebige Koalition.
  \begin{equation}
    v(K) := \max_{B'\in B(K)}(v(B'))
  \end{equation}
\end{definition}

Insbesondere gilt dann auch für zwei Koalitionen K,S:
\begin{equation}
  v(K\cup S) = \max_{B'\in B(K\cup S)}(v(B'))
\end{equation}

Beachte auch das für einen Outcome zweier Koalitionen eine mögliche Baustellenmenge der potentiellen Baustellen der vereinigung der Koalitionen exestiert die alle Baustellen des seperaten Outcomes beinhaltet:
\begin{equation}
  \forall (B_K, B_S)\in Outcome(K,S).\exists B'\in B(K\cup S).B_K\subset B'\land B_S\subset B'
\end{equation}

Hierraus können wir die Superadditivität des Spiels schließen:
\begin{equation}
  v(K\cup S) \geq v(K) + v(S)
\end{equation}


Insbesondere ist das Spiel Konvex.
Nach einem Theorem (TODO) ist bei einem Konvexen Spiel das Shapley Value im Core enthalten sodass die Beste Lösung dieses Spiels die große Koalition ist und eine Stabile und faire lösung immer exestiert.





% \section{Vorgehen}
% Zunächst werden wir versuchen unterschiedliche optimierungskritärien (utility) zu formulieren wie z.B. Optimierung der Gewinne aller Agenten. Oder optimieren der Gewinne bei gleichzeitiger minimierung der Kosten der Baustellen.
%
%
% \subsection*{Analysekritärien}
% Die Analysetechnik besteht nun darin folgende Fragen zu Formalisieren und gegebene Modelle darauf zu untersuchen:
%
% \begin{enumerate}
%   \item $\varphi_{\text{Optimal}}\Leftrightarrow$ Es exestiert kein Matching, das bez. der optimalitätskritärium besser währe.
%   \item $\varphi_{\text{pareto-effizient}}\Leftrightarrow$ Kein Spieler kann sich durch Manipulation seines Matchings verbessern.
%   \item Existenz von dummy und veto spielern
%   \item eindeutigkeit des optimums
% \end{enumerate}

% \begin{dot2tex}[dot,mathmode]
% graph G {
%       splines=false;
%       node[shape=circle, style=filled]
%       subgraph cluster_1 {
%       subgraph cluster_1r {
%          a12 [label="a"]
%          b12 [label="b"]
%          c12 [label="c"]
%          d12 [label="d"]
%          e12 [label="e"]
%          a12--b12--c12--d12--e12 [style=invis]
%          }
%       subgraph cluster_1l {
%          a11 [label="a"]
%          b11 [label="b"]
%          c11 [label="c"]
%          d11 [label="d"]
%          e11 [label="e"]
%          a11--b11--c11--d11--e11 [style=invis]
%          }
%          c11--a12 [constraint=false]
%          c11--b12 [constraint=false]
%          d11--b12 [constraint=false]
%          e11--a12 [constraint=false]
%          e11--b12 [constraint=false]
%      }
% }
% \end{dot2tex}


\end{document}
