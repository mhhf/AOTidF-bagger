% Article template for Mathematics Magazine
% Revised 7/2002  Thanks for Greg St. George
\documentclass[12pt]{article}
\usepackage{amssymb}
\usepackage[ngerman]{babel}
\usepackage[utf8]{inputenc}
\usepackage{amsmath}
\usepackage{amsthm}
\usepackage{graphicx}
\usepackage{tikz}
\usepackage{dot2texi}
\usetikzlibrary{shapes,arrows}
\renewcommand{\baselinestretch}{1.2}
%This is the command that spaces the manuscript for easy reading


%todo
\usepackage[colorinlistoftodos,prependcaption,textsize=tiny]{todonotes}
\usepackage{xargs}                      % Use more than one optional parameter in a new commands
\newcommandx{\QUESTION}[2][1=]{\todo[linecolor=none,backgroundcolor=blue!15,bordercolor=none,#1]{\textbf{QUESTION: }#2}}


\newtheoremstyle{break}
  {\topsep}{\topsep}%
  {\itshape}{}%
  {\bfseries}{}%
  {\newline}{}%
\theoremstyle{break}
\newtheorem{zeige}{Zeige}
\newtheorem{definition}{Definition}
\newtheorem{bsp}{Beispiel}



\begin{document}
%\thispagestyle{empty}
\title{AOTidf Bagger Game}
\author{Denis Erfurt, Tobias Behrens}
\maketitle

\section{Formalisierung des Problems}

\begin{definition}
  Eine Coalitioal Skill Game Signatur \\$\sigma_{CSG}:=\sigma_{Ar}\cup\{Agent_{/1}, Baustelle_{/1}, Supply_{/3}, Demand_{/3}, Budget_{/2}, kosten_{/3\mapsto 1}, M_{/4} \}$. Dabei steht $\sigma_{Ar}$ für die Signatur der Standardarithmetik mit $0,1,+,*\in\sigma_{Ar}$
    $Agent(x) :\Leftrightarrow\text{ x ist ein Agent}$ \\
    $Baustelle(x) :\Leftrightarrow\text{ x ist eine Baustelle}$ \\
    $Supply(t, x, y) :\Leftrightarrow\text{ Agent x besitzt y Einheiten vom typ t}$ \\
    $Demand(t, x, y) :\Leftrightarrow\text{ Baustelle x benötigt y Einheiten vom typ t}$ \\
    $Budget(x, n) :\Leftrightarrow$
    Baustelle x ist maximal bereit einen Gewin von n bei fertigstellung auszuzahlen \\
    $kosten(t, n, x, y):\Leftrightarrow$ Funktion die die Kosten für einen Agent x für den Transport n Ressourcen t an die Baustelle y berrechnet.\\
    $M(x, t, n, y, v) :\Leftrightarrow$ Agent x sendet n Ressourcen des Types t an die Baustelle y und bekommt die Vergütung v
\end{definition}

\begin{bsp}
  Sei $\mathcal{A}$ eine $\sigma_{CSG}-$Struktur:\\
  $Agent^\mathcal{A} := \{a_1, a_2, a_3\}$ \\
  $Baustelle^\mathcal{A} := \{b_1, b_2\}$ \\
  $Supply^\mathcal{A} := \{
  (t_1, a_1, 2),
  (t_2, a_1, 7),
  (t_1, a_2, 3),
  (t_2, a_2, 5),
  (t_1, a_3, 20),
  (t_2, a_3, 5)
  \}$ \\
  $Demand^\mathcal{A} := \{
  (t_1, b_1, 10),
  (t_2, b_1, 5),
  (t_1, b_2, 2),
  (t_2, b_2, 2)
  \}$ \\
  $Budget^\mathcal{A}:= \{
  (b_1, 10),
  (b_2, 3)
  \}$ \\
  $M := \{
  (a_1, t_1, 2, b_1),
  (a_1, t_2, 5, b_1),
  (a_2, t_1, 3, b_1),
  (a_3, t_1, 5, b_1),
  (a_3, t_1, 2, b_2),
  (a_3, t_2, 2, b_2)
  \}$
\end{bsp}

\section{Vorgehen}
Zunächst werden wir versuchen unterschiedliche optimierungskritärien (utility) zu formulieren wie z.B. Optimierung der Gewinne aller Agenten. Oder optimieren der Gewinne bei gleichzeitiger minimierung der Kosten der Baustellen.


\subsection*{Analysekritärien}
Die Analysetechnik besteht nun darin folgende Fragen zu Formalisieren und gegebene Modelle darauf zu untersuchen:

\begin{enumerate}
  \item $\varphi_{\text{Optimal}}\Leftrightarrow$ Es exestiert kein Matching, das bez. der optimalitätskritärium besser währe.
  \item $\varphi_{\text{pareto-effizient}}\Leftrightarrow$ Kein Spieler kann sich durch Manipulation seines Matchings verbessern.
  \item Existenz von dummy und veto spielern
  \item eindeutigkeit des optimums
\end{enumerate}

% \begin{dot2tex}[dot,mathmode]
% graph G {
%       splines=false;
%       node[shape=circle, style=filled]
%       subgraph cluster_1 {
%       subgraph cluster_1r {
%          a12 [label="a"]
%          b12 [label="b"]
%          c12 [label="c"]
%          d12 [label="d"]
%          e12 [label="e"]
%          a12--b12--c12--d12--e12 [style=invis]
%          }
%       subgraph cluster_1l {
%          a11 [label="a"]
%          b11 [label="b"]
%          c11 [label="c"]
%          d11 [label="d"]
%          e11 [label="e"]
%          a11--b11--c11--d11--e11 [style=invis]
%          }
%          c11--a12 [constraint=false]
%          c11--b12 [constraint=false]
%          d11--b12 [constraint=false]
%          e11--a12 [constraint=false]
%          e11--b12 [constraint=false]
%      }
% }
% \end{dot2tex}


\end{document}
