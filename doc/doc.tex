% Article template for Mathematics Magazine
% Revised 7/2002  Thanks for Greg St. George
\documentclass[sigconf]{acmart}
% \documentclass[manuscript]{acmart}
\usepackage{amssymb}
\usepackage[ngerman]{babel}
\usepackage[utf8]{inputenc}
\usepackage{amsmath}
\usepackage{amsthm}
\usepackage{graphicx}
\usepackage{tikz}
\usepackage{tikz-cd}
\usepackage{dot2texi}
\usetikzlibrary{shapes,arrows}
\renewcommand{\baselinestretch}{1.2}
%This is the command that spaces the manuscript for easy reading

\setlength\parindent{0pt}

%todo
\usepackage[colorinlistoftodos,prependcaption,textsize=tiny]{todonotes}
\usepackage{xargs}                      % Use more than one optional parameter in a new commands
\newcommandx{\QUESTION}[2][1=]{\todo[linecolor=none,backgroundcolor=blue!15,bordercolor=none,#1]{\textbf{QUESTION: }#2}}
\newcommandx{\TODO}[2][1=]{\todo[inline,linecolor=none,backgroundcolor=blue!15,bordercolor=none,#1]{\textbf{TODO: }#2}}

\newtheoremstyle{break}
  {\topsep} {\topsep}%
  {}{}%
  {\bfseries}{:}%
  {\newline}{}%
\theoremstyle{break}
\newtheorem{zeige}{Zeige}
% \newtheorem{definition}{Definition}
\newtheorem{bsp}{Beispiel}
\newtheorem{thm}{Theorem}
\newtheorem{deff}{Definition}

\setcopyright{none}

\begin{document}
%\thispagestyle{empty}
\title{AOTidf Bagger Game}
\author{Denis Erfurt, Tobias Behrens, Abdallah Kadour}

\begin{abstract}
  \noindent
  \TODO{Abstract}
  % In dieser Arbeit zeigen wir, dass es sich bei dem \textsc{Coalition Skill Game} (CSG) um ein superadditives Spiel handelt. Bei einem superadditives Spiel ist die große Koalition zwangsläufig die beste Strategie, um den Gewinn aller Agenten zu maximieren. Um diesen Zusammenhang aufzuzeigen, formalisieren wir zunächst das in der Aufgabenstellung beschriebene Spiel: So können wir zeigen, dass jedes CSG bezüglich seiner Strategien equivalent ist zu einem Spiel, in dem jeder Agent nur über eine Einheit eines Skilltyps verfügt. Anschließend zeigen wir die Superadditivitätseigenschaft für das Spiel und geben ein Algorithmus an, der die optimale Ressourcenverteilung aus der Sicht der Agenten berechnet. Außerdem geben wir einen verteilten Algorithmus an, der Agenten und Baustellen eine möglichst optimalen Lösung über gegenseitige Aushandlungen in Gestalt von Auktionen ermöglicht.
\end{abstract}


\maketitle

\section{Gliederung}
\TODO{gliederung}

\section{Voraussetzungen}
In diesem Kapitel geben wir in Abschnitt (\ref{task}) einen Gesamtüberblick über die Aufgabenstellung und formulieren Prämissen für die weitere Bearbeitung. Im Abschnitt (\ref{basics}) werden notwendige theoretische Grundlagen zusammengefasst, die für das weitere Verständnis der Arbeit notwendig sind.

\subsection{Aufgabenstellung}
\label{task}
Als gegebenes Szenario war eine Menge an Bauaufträgen und Baufirmen, im folgenden Agenten genannt, gegeben. Den Agenten verfügen über ein Bestand an Skillkapazitäten und Baustellen besitzen ein Bedarf an Skillkapazitäten der für die Ausschüttung eines Erlöses gedeckt werden muss.

Diese Arbeit Modelliert (Kapitel \ref{mod}) dieses Szenario und untersucht Mechanismen (Kapitel \ref{ergebnisse}), die mit hilfe von Koalitionsbildung eine Zuordnung von Bestand und Bedarf an Skillkapazitäten berechnen, bezüglich der \textbf{sozialen Wohlfahrt, Stabilität und Fairness} der Gewinnausschüttung sowie ihrer \textbf{absoluten Performance}.

\subsubsection*{Prämissen}
Allgemein gehen wir von folgenden Grundannahmen aus:

\begin{enumerate}
  \item \textbf{Rationalität}: Agenten arbeiten ausschließlich für ihr eigenes Interesse.
  \item \textbf{Multiskill}: Agenten können mehrere Skilltypen mit beliebiger quantität besitzen.
  \item \textbf{linearität}: Skillkapazitäten können höhstens ein mal eingesetzt werden und werden nach ihrem einsatz "verbraucht".
  \item \textbf{unvollständige Information der Konkurenz}: Agenten haben keine Information über die Ressourcen anderer Agenten.
  \item \textbf{vollständige Informationen des Bedarfs}: Agenten haben vollständige Information über die Anzahl, Position, Bedarf sowie Vergütung der Bauaufträge.
  \item \textbf{Zeitagnostisch}: Alle betrachtungen werden ohne Zeit gemacht: insbesondere verändert sich nichts an den Bauaufträgen oder den Agenten.
\end{enumerate}

\subsection{Grundlagen}
\label{basics}
  Im weiteren Verlauf der Arbeit werden Funktionen mit kleinem Anfangsbuchstaben und Relationen mit einem großen Anfangsbuchstaben geschrieben. Ebenfalls gehen wir davon aus, dass alle Funktionen total und mit 0 initialisiert sind, falls für eine Struktur Eingabeparameter nicht näher definiert wurden. Wir verwenden die Schreibweise in Prädikatenlogik und die Mengenschreibweise equivalent: $a\in A \equiv A(a)$
  Um das gegebene Problem strukturell analysieren zu können, benutzen wir die Sprache der HOL (Higher Order Logic).


%%%%%%%%%%%%%%%%%%%%%%%%%%%%%%%%%%%%%%%%
%%%%%%%%%%%%%%%%%%%%%%%%%%% MODELLIERUNG
%%%%%%%%%%%%%%%%%%%%%%%%%%%%%%%%%%%%%%%%
\section{Modellierung}
\label{mod}
In diesem Kapitel verfolgen wir das Ziel, durch eine Formalisierung der Aufgabenstellung eine geeignete Grundlage zu für die Formulierung von Mechanismen und die Beantwortung zentraler Fragestellungen zu schaffen. Zunächst werden wir in Abschnitt (\ref{sigmod}) grundlegende Definitionen vorstellen sowie in Abschnit (\ref{sigandmod}) zwei Signaturen sowie die dazugehörigen Modellklassen einführen.

\subsection{Signaturen und Modellklassen}
\label{sigandmod}
Wir führen nun zwei verschiedene Signaturen mit den dazugehörigen Modellklassen ein:
\begin{enumerate}
  \item \textbf{CSGS} - Coalition Skill Game Setting
  \item \textbf{CSG} - Coalition Skill Game
\end{enumerate}

\begin{definition}[CSGS]
  Eine \textsc{Coalitional Skill Game Setting}-Signatur (CSGS-Signatur) sei definiert als
  \begin{align*}
    &\sigma_{CSGS}:= \\
    &\{Agent_{/1}, Baustelle_{/1}, supply_{/2}, demand_{/2}, budget_{/1}, kosten_{/4} \}
  \end{align*}
  \textbf{Intuition} \\
    \begin{tabular}{lrl}
    $Agent(x)$&$:\Leftrightarrow$& $x$ ist ein Agent (Baufirma) \\
    $Baustelle(x) $&$:\Leftrightarrow$& $x$ ist eine Baustelle \\
    $supply(x, t)\mapsto n $&$:\Leftrightarrow$& Agent $x$ besitzt $n$ Einheiten \\&& vom Skilltyp $t$ \\
    $demand(x, t)\mapsto n $&$:\Leftrightarrow$& Baustelle $x$ benötigt $n$ Einheiten \\&& vom Skilltyp $t$ \\
    $budget(x)\mapsto n $&$:\Leftrightarrow$&
    Baustelle $x$ zahlt einen Gewinn $n$ \\&& bei Fertigstellung aus\\
    $kosten(t, n, x, y)\mapsto n$&$:\Leftrightarrow$& Kosten für Agenten $x$ für die \\&& Bereitstellung von $n$ Einheiten des \\&& Skilltyp $t$ an Baustelle $y$.
    \end{tabular}
\end{definition}

Im folgenden werden wir die $\sigma_{CSGS}$-Struktur $\mathcal{S}$ als ein Setting bezeichnen. Die Modellklasse $M_{CSGS}$ steht für die Gesamtheit an validen Settings, die zudem die folgende Bedingung erfüllt:
\begin{align}
  Agent\cap Baustelle = \emptyset
\end{align}

\begin{bsp}[]
  Ein Beispiel für eine valides Setting ist die Struktur $\mathcal{S}\in M_{CSGS}$ mit:\\

  \setlength{\tabcolsep}{24pt}
  \begin{tabular}{l|l}
    $Agent^\mathcal{S} := \{a_1, a_2, a_3\}$ & $Baustelle^\mathcal{S} := \{b_1, b_2\}$ \\
    $supply^\mathcal{S}(a_1, t_1)\mapsto 2 $ & $demand^\mathcal{S}(b_1, t_1)\mapsto 10$\\
    $supply^\mathcal{S}(a_1, t_2)\mapsto 7 $ & $demand^\mathcal{S}(b_1, t_2)\mapsto 5 $\\
    $supply^\mathcal{S}(a_2, t_1)\mapsto 3 $ & $demand^\mathcal{S}(b_2, t_1)\mapsto 2 $\\
    $supply^\mathcal{S}(a_2, t_2)\mapsto 5 $ & $demand^\mathcal{S}(b_2, t_2)\mapsto 2 $\\
    $supply^\mathcal{S}(a_3, t_1)\mapsto 20$ & $budget^\mathcal{S}(b_1)\mapsto 10$\\
    $supply^\mathcal{S}(a_3, t_2)\mapsto 5 $ & $budget^\mathcal{S}(b_2)\mapsto 3 $\\
  \end{tabular}
\end{bsp}

\begin{definition}[CSG]
  Eine \textsc{Coalitional Skill Game}-Signatur (CSG-Signatur) sei definiert als
  \begin{align*}
    \sigma_{CSG}:=\sigma_{CSGS}\cup\{m_{/3}, v_{/2}\}
  \end{align*}
  \textbf{Intuition} \\
    \begin{tabular}{lrl}
    $m(x, t, y)\mapsto n$&$:\Leftrightarrow$& Agent $x$ sendet $n$ Einheiten des Skilltyps $t$ \\&& an die Baustelle $y$ \\
    $v(x,y)\mapsto n$&$:\Leftrightarrow$& Agent $x$ erhält von Baustelle $y$ \\&& die Vergütung $n$
    \end{tabular}
\end{definition}

Wir bezeichnen eine $\sigma_{CSG}$-Struktur als ein Game $\mathcal{G}$. Ein Game ist also ein valides Setting mit einer validen Lösung: eine mögliche Zuordnung verfügbarer Skills von Baufirmen zu Baustellen, die zu einer definierten Erlösausschüttung der Baustellen führt.

$M_{CSG}$ bezeichnen wir als Klasse aller valider Games, die folgende (informale) Eigenschaften erfüllen:
\begin{enumerate}
  \item Ein Agent kann nicht mehr Einheiten eines Skills ausgeben als er besitzt.
  \item Eine Baustelle schüttet genau dann ihr Geld aus, wenn sie vollständig mit den benötigten Skills versorgt wird.
\end{enumerate}

% \begin{bsp}
%   Sei $\mathcal{G}$ eine Erweiterung der $\sigma_{CSG}$-Struktur $\mathcal{S}$ mit: \\
%
%   \setlength{\tabcolsep}{24pt}
%   \begin{tabular}{l|l}
%     $m(a_1, t_1, b_1) \mapsto 2 $&$ m(a_1, t_2, b_1) \mapsto 5$ \\
%     $m(a_2, t_1, b_1) \mapsto 3 $&$ m(a_3, t_1, b_1) \mapsto 5$ \\
%     $m(a_3, t_1, b_2) \mapsto 2 $&$ m(a_3, t_2, b_2) \mapsto 2$
%   \end{tabular}
%   \\
%   \TODO{vergütung und veranschaulichung}
% \end{bsp}
%
Sei $M_{OCSG} \subset M_{CSG}$ eine Modellklasse die zusätzlich ''optimal'' ist:
Es existiert kein Matching, bei dem der summierte Gewinn über alle Agenten größer ist. Sie enthält so alle Zuordnungen, die die soziale Wohlfahrt maximieren.

Sei $M_{FOCSG}\subset M_{OCSG}$ eine Modellklasse, die zusätzlich ''fair'' ist: Die Erlösverteilung an die Agenten entspricht damit in unserem Verständnis dem Shapley-Value: Jeder Agent erhält den Durchschnitt seiner marginalen Beiträge zu jeder möglichen Permutation der Reihenfolge der Spieler. Der marginale Beitrag eines Spielers $a$ zu einer Reihenfolge (betrachtet als Koalition\footnote{näher definiert im nächsten Abschnitt}) ist definiert als Differenz des Gewinns zwischen der Koalition aus allen Agenten vor $a$ in der Reihenfolge und der Koalition aus allen Agenten vor $a$ und mit $a$ in der Reihenfolge.

Insbesondere interessieren wir uns hier für die Frage, ob eine faire Verteilung existiert: $M_{FOCSG} =^? \emptyset$

\subsection{Terminologie}
\label{sigmod}

\begin{definition}[Koalition]
  Eine \textbf{Koalition} ist ein Zusammenschluss von Agenten, die vollständige Informationen über alle Koalitionsteilnehmer besitzen und die Verteilung gemeinsamer Ressourcen auf Bauaufträge beabsichtigen. Ebenfalls haben sie einen gemeinsamen Mechanismus, welcher den Erlös auf die Koalitionsteilnehmer aufteilt. Formal: $K\subseteq Agenten$ mit folgenden Eigenschaften:
  \begin{align}
    \exists x\exists t\exists y. m(x,t,y) > 0 \land K(x)&\rightarrow (\forall x'\forall t' m(x',t',y) > 0\Rightarrow K(x')) \label{onebaustelle} \\
    \exists x\exists y v(x,y) > 0 \land K(x) &\rightarrow (\forall x'.v(x', y) > 0 \rightarrow K(x')) \label{oneprofit}
  \end{align}
  Dabei sagt Bedingung (\ref{onebaustelle}) aus, dass eine Baustelle nur von einer Koalition gebaut werden kann und Bedingung \ref{oneprofit}, dass eine Baustelle nur Geld an eine Koalition senden kann.
\end{definition}

\begin{definition}[Matching]
  Ein \textbf{Matching} ist eine konkrete Zuordnung von Skillkapazitäten zu Bauaufträgen.
  Formal das Prädikat $m\in\sigma_{CSG}$.
\end{definition}

% \begin{definition}[Gewinnverteilung]
%   Eine \textbf{Gewinnverteilung} ist eine Relation, die für eine konkrete Koalition mit einem Konkreten matching sowie dem dazu verbundenen Erlös eine Auszahlungsverteilung an jeden Koalitionsteilnehmer berechnet.
% \end{definition}


\subsection{Mechanismen}
\TODO{Mechanismuskriterien}
\TODO{Phasen}
\TODO{überarbeiten}

Bei der gegebenen Aufgabenstellung interessieren wir uns für ein Mechanismus, der bei einem gegebenen Setting ein optimales Game mit einer fairen Verteilung dezentral hervorbringt. Formal gesehen:

\begin{equation}
  m: M_{CSGS} \rightarrow M_{FOCSG}
\end{equation}
% \TODO{define mechnismus kritärien}
% Die Kriterien, nach denen der dezentrale Mechanismus arbeitet und das Ergebnis -- bei uns ein Game -- bewertet wird, ist nicht hart formuliert und lässt uns Gestaltungspielraum. Die Forderung an das Ergebnis ist, dass die ''soziale Wohlfahrt maximiert wird und die Gewinne möglichst stabil und fair verteilt werde''.



%%%%%%%%%%%%%%%%%%%%%%%%%%%%%%%%%%%%%%%%
%%%%%%%%%%%%%%%%%%%%%%%%%%%%% ERGEBNISSE
%%%%%%%%%%%%%%%%%%%%%%%%%%%%%%%%%%%%%%%%
\section{Ergebnisse}
\label{ergebnisse}
\subsection{Theoretische Ergebnisse}

\subsubsection{Superadditivität}

\label{supadd}
Im Folgenden zeigen wir die Superadditivität das CSG. Intuitiv zeigen wir, dass ein Spieler immer nur einen positiven Wert in eine Koalition einbringt.

\begin{lemma}[Superadditivität]
Das CSG ist superadditiv.
\end{lemma}

Bevor wird die Superadditivität des CSG zeigen, führen wir zunächst zwei weitere ''simple'' Signaturen mit dazugehörigen Modellklassen ein und zeigen, dass Ergebnisse auf ihnen direkt auf das CSG übertragen werden können. Außerdem führen wir die Begriffe \textit{potentielle Matches} einer Koalition, die Menge der \textit{gleichzeitig möglichen Baustellen einer Koalition}, die \textit{Outcome-Menge} von zwei Koalitionen und den \textit{Wert} einer Baustelle ein.

\begin{enumerate}
  \item \textbf{SCSGS} - Simple Coalition Skill Game Setting
  \item \textbf{SCSG} - Simple Coalition Skill Game
\end{enumerate}

\noindent
Wir zeigen im folgenden, dass sich jedes Setting bzw. Game zu einem Setting bzw. Game vereinfachen lässt, in dem jeder Agent nur eine Einheit eines Skilltypen besitzt. Diese Vereinfachung erleichtert uns den Beweis der Superadditivität und die Betrachtung möglicher Algorithmen zur Verteilungsberechnung. Hierfür benötigen wir jedoch die neue Relation $AgentOwner_{/2}$, mit der wir uns die Zuteilung einer Skillkapazität zu seinem ursprünglichen Agenten merken:

\begin{definition}[SCSGS]
  Eine \textsc{Simple Coalition Skill Game Setting}-Signatur sei definiert als
  \begin{align*}
    \sigma_{SCSGS}:= &\{gentOwner_{2}, Agent_{/1}, Baustelle_{/1}, supply_{/2}, \\
    &demand_{/2}, budget_{/1}, kosten_{/4} \}
  \end{align*}
  \noindent
  \textbf{Intuition} \\
    \begin{tabular}{lrl}
    $AgentOwner(a, x)$&$:\Leftrightarrow$& Agent $x$ gehört zu dem Agenten $a$ \\
    $Agent(x)$&$:\Leftrightarrow$& $x$ ist ein Agent (Baufirma) \\
    $Baustelle(x) $&$:\Leftrightarrow$& $x$ ist eine Baustelle \\
    $type(x)\mapsto t $&$:\Leftrightarrow$& Agent $x$ ist vom Skilltyp $t$ \\
    $demand(x, t)\mapsto n $&$:\Leftrightarrow$& Baustelle $x$ benötigt $n$ Einheiten \\&& vom Skilltyp $t$ \\
    $budget(x)\mapsto n $&$:\Leftrightarrow$&
    Baustelle $x$ zahlt einen Gewinn $n$ \\&& bei Fertigstellung aus\\
    $kosten(t, n, a, y)\mapsto n$&$:\Leftrightarrow$& Kosten für zugehörigen  Agenten $a$ \\&& für die Bereitstellung von $n$ Einheiten \\&& des Skilltyps $t$ an Baustelle $y$
    \end{tabular}
\end{definition}

\noindent
Wir bezeichnen eine $\sigma_{SCSGS}$-Struktur $\mathcal{SS}$ als \textit{Simple Setting} und die dazugehörige Modellklasse valider Strukturen $M_{SCSGS}$. Validitätskritärien sind analog zu $M_{CSGS}$.

\begin{definition}[SCSG]
  Eine \textsc{Simple Coalition Skill Game}-Signatur (SCSG) sei definiert als
  \begin{align*}
    \sigma_{SCSG}:=\sigma_{SCSGS}\cup\{M_{/3}, v_{/2}\}
  \end{align*}
  \noindent
  \textbf{Intuition} \\
    \begin{tabular}{lrl}
    $m(x, t, y)$&$:\Leftrightarrow$& Agent $x$ sendet eine Einheiten des Skilltyps $t$ \\&& an die Baustelle $y$ \\
    $v(x,y)\mapsto n$&$:\Leftrightarrow$& Agent $x$ erhält von Baustelle $y$ die Vergütung $n$
    \end{tabular}
\end{definition}

\noindent
Wir bezeichnen eine $\sigma_{SCSG}$-Struktur $\mathcal{SG}$ als \textit{Simple Game} mit der dazugehörigen validen Modelklasse $M_{SCSG}$. Validitätskriterien sing analog zu $M_{CSG}$.


\subsection{Beziehungen zwischen den Modellklassen}
\label{bez}

Die nachfolgenden Betrachtungen wollen wir jedoch auf der vereinfachten Strukturen anstellen: Bei  dieser besitzt jeder Agent nur eine Einheit eines Skilltyps. Um dennoch Aussagen über das CSG machen zu können, zeigen wir, dass sich jedes Setting bzw. Game zu einem Simple Setting bzw. Simple Game überführen lässt. Das erlaubt uns die Ergebnisse von theoretische Betrachtungen von Mechanismen, die bei einem Simple Setting ein Simple Game berechnen, auf das CSG zu übetragen.

\noindent
Formal:
\[
\begin{tikzcd}[column sep=1in,row sep=1in]
M_{CSGS} \arrow{d}{\pi} \arrow{r}{\pi'^{-1}\ \circ\  m'\ \circ\ \pi} & M_{CSG} \\
M_{SCSGS} \arrow{r}{m'} & M_{SCSG} \arrow{u}{\pi'^{-1}}
\end{tikzcd}
\]

\noindent
Dabei müssen folgende Eigenschaften gelten:
\begin{eqnarray}
  \pi &-&\text{total, injektiv} \\
  \pi^{-1}&-&\text{surjektiv} \\
  \pi' &-&\text{total, injektiv} \\
  \pi'^{-1}&-&\text{surjektiv} \\
  \pi^{-1}\circ\pi &=& id_{M_{CSGS}} \\
  \pi'^{-1}\circ\pi' &=& id_{M_{CSG}}
\end{eqnarray}

\noindent
Weiter werden wir nur Mechanismen betrachten, die am Setting keine Änderungen vornehmen, sondern nur Matchings und die Vergütungsverteilung bestimmen. Auf eine ausführliche Definition der gesuchten Funktionen wird hier ebenfalls verzichtet,  stattdessen wird eine Intuition gegeben: Ein Setting bzw. Game lässt sich in ein Simple Setting bzw. Simple Game überführen, indem jede Einheit eines Skilltyps eines Agenten als eigenständiger Agent betrachtet wird.

Dabei wird die Zugehörigkeit von Agenten zum Skill in der $Agent\-Owner$ Relation gesichert. Die Kostenfunktion bleibt bestehen und wird lediglich auf den $AgentOwner$ übertragen. Bei den Matches und Vergütungen wird analog verfahren.

\begin{definition}[Potentielle Matches einer Koalition]
  Sei $K\subseteq Agent$ eine beliebige Koalition. Für $K$ definieren wir die \textit{Menge der potentiellen Matches} $PM(K)$ als:

  \begin{eqnarray}
    PM(K) := \{ & m: Agent \times Skilltyp \times Baustelle \rightarrow \mathbb{N}\ |\\
    & \phi_{Match}(m) \land \\
    & \quad \quad \forall x\;\forall t\;\forall y: m(x,t, y) > 0 \rightarrow x\in K \quad \quad\} \label{matchown}
  \end{eqnarray}

  Dabei bezeichnet $\phi_{Match}(m)$ die Menge der validen Matches:
  \begin{eqnarray}
    \phi_{Match}(m) := \forall x\; \forall t\; : \left(\;\sum_{y\in Baustelle} m(x,t,y)\;\right)\;\leq \;supply(x,t)
  \end{eqnarray}
  Der Ausdruck (\ref{matchown}) verdeutlicht, dass nur solche Matches für eine Koalition betrachtet werden, deren liefernder Agent auch Teil der Koalition ist.
\end{definition}

\begin{definition}[Gleichzeitig mögliche Baustellen einer Koalition]
  Sei $K\subseteq Agent$ eine beliebige Koalition. Die Menge der möglichen Baustellen, die von $K$ gleichzeitig gebaut werden können, $B(K)$, ist definiert als:

  \begin{align}
     B(K) := \{\quad &B_k \subseteq Baustelle\ | \\
     &\exists m\in PM(K), \;\forall b\in B_k, \;\forall t: \\
     &\left(\; \sum_{x\in K}m(x,t,b) \;\right)\geq demand(b, t) \quad\}
  \end{align}

\end{definition}

\noindent
Da wir ein Simple Game betrachten, Können wir o.B.d.A. annehmen, dass eine Agent nur einer Koalition zugeordnet ist:
\begin{equation}
  K\neq S \Rightarrow K\cap S =\emptyset \label{koalitiondisjunct}
\end{equation}

\noindent
Hieraus folgt, dass eine Baustelle in einem Spiel nur von einer Koalition gebaut werden kann, da eine Baustelle pro Spiel nur einmal gebaut wird. Um dies formal festzuhalten, definieren wir im Folgenden die Outcome-Relation:

\begin{definition}[Outcome-Menge zweier Koalitionen]
  Seien $K, S \subseteq Agent$ beliebige Koalitionen. Eine \textit{Outcome-Menge} weist beiden Koalitionen ein Tupel der Baustellen zu, die sie jeweils zur gleichen Zeit bauen können:
\begin{align}
  Outcome(K,S) := \{ \; \;
  &(B_K,B_S)\subseteq Baustelle\times Baustelle\ | \\
  B_K &\in B(K) \land B_S\in B(S)\land B_K\cap B_S =\emptyset \;\;\}
\end{align}
\end{definition}

\noindent
% Zur weiteren Vereinfachung definieren wir noch eine Bewertungsfunktion für eine Menge von Baustellen:
%
% \begin{definition}[Wert einer Baustelle]
%   Sei $B\subseteq Baustelle$ beliebig. Dann ist der \textit{Wert} von $B$ definiert als
%   \begin{equation}
%     v(B):=\sum_{b \in B} \; budget(b)
%   \end{equation}
% \end{definition}
%
% \begin{definition}[erzielbarer Wert zweier Koalitionen]
%   Seien $K, S \subseteq Agent$ beliebige Koalitionen. Der \textit{erzielbare Wert zweier Koalitionen} $K$ und $S$ ist nun definiert als
% \begin{align}
%   v(K) + v(S) := \max_{(B_K, B_S)\in Outcome(K,S)} \;(\;
%   &v(B_K) - \varphi_{kosten}(K, B_K) \;+ \\ &v(B_S) - \varphi_{kosten}(S, B_S) \qquad)
% \end{align}
%
% \noindent
% Die Baustellen, die von einer Koalition gebaut werden, können nicht mehr von einer anderen Koalition gebaut werden. Deshalb müssen wir diese zeitgleich betrachten.
% $\varphi_{kosten}(B,K)$ ist dabei diejenige Funktion, die basierend auf der vorhandenen Kostenfunktion, der Koalition und den Baustellen die Kosten der Koalition für die Bereitstellung aller notwendigen Skills berechnet, um die Baustellen fertig zu stellen.
% \textbf{Bemerkung}: Nach Definition ist das Ergebnis pareto-effizient.
% \end{definition}
%
% \noindent
% Existiert eine Koalition isoliert, lässt sich diese auch isoliert betrachten:
% \begin{definition}
%   Sei $K \subseteq Agent$ eine beliebige Koalition.
%   \begin{equation}
%     v(K) := \max_{B'\in B(K)} \left(\; v(B') - \varphi_{kosten}(B', K) \;\right)
%   \end{equation}
% \end{definition}
%
% \noindent
% Insbesondere gilt dann auch für zwei allein agierende Koalitionen $K, S \subseteq Agent$:
% \begin{equation}
%   v(K\cup S) = \max_{B'\in B(K\cup S)} \left(\; v(B')-\varphi_{kosten}(B', K) \;\right)
% \end{equation}
%
% \noindent
Beachte auch das für einen Outcome zweier Koalitionen eine mögliche Baustellenmenge der potentiellen Baustellen der Vereinigung der Koalitionen existiert die alle Baustellen des separaten Outcomes beinhaltet:
\begin{equation}
  \forall (B_K, B_S)\in Outcome(K,S),\; \exists B'\in B(K\cup S):\; B_K\cup B_S\subseteq B'
\end{equation}

\noindent
Hieraus können wir die Superadditivität des Spiels schließen:
\begin{equation}
  v(K\cup S) \geq v(K) + v(S)
\end{equation}
\begin{flushright}
  \qed
\end{flushright}

\subsubsection{Instabilität der großen Koalition}
\label{instabil}

\begin{lemma}[Instabilität der großen Koalition]
  Im allgemeinen Fall ist die große Koalition $K=Agenten$ instabil.
\end{lemma}

Beweis durch Beispiel:
\begin{figure}
  \centering
  \includegraphics[width=0.5\textwidth]{example-exchangeable-agents.png}
  \caption{Szenario mit einer instabilen großen Koalition.}
  \label{szenario1}
\end{figure}

Im Szenario in der Abbildung (\ref{szenario1}) ist der Gewinn der Koalitionen $K_1 = \{A0, A1, A2\}$ und $K_2 = \{A0, A2\}$ gleich, jedoch erhält die Agenten $A0$ in $K_2$ anteilig mehr vom Gewinn. Demnach lohnt es sich für ihn die große Koalition zu verlassen.
\begin{flushright}
  $\qed$
\end{flushright}

\subsubsection{NP-Härte des Problems}
\label{np}
Im folgenden zeigen wir die NP-Härte eines gewünschten Mechanismus. Hierfür reduzieren wir das Rucksackproblem, ein allgemein bekanntes NP-vollständiges Problem \cite{Karp1972}, auf ein CSG.
Das Rucksackproblem ist folgendermaßen definiert:

\begin{lemma}
Sei $m:M_{CSGS}\rightarrow M_{OCSG}$ ein Mechanismus, der ein optimales Matching berechnet, dann ist $m$ NP-hart.
\end{lemma}

\begin{definition}[KNAPSACK]
Sei $U$ eine Menge von Objekten, $w$ eine Gewichtsfunktion und $v$ eine Nutzenfunktion:
\begin{align}
  w: U\rightarrow \mathbb{R} \\
  v: U\rightarrow \mathbb{R}
\end{align}
Sei weiter $B\in\mathbb{R}$ eine vorgegebene Gewichtsschranke.
Gesucht ist eine Teilmenge $K\subseteq U$, die folgende Bedingungen erfüllt:
\begin{align}
  \sum_{w\in K}w(u)\leq B \label{bed1}\\
  max(\sum_{w_in K}v(u)) \label{bed2}
\end{align}
\end{definition}

Sei $K=\{U,w,v\}\in KNAPSACK$ ein beliebiges Rucksackproblem. Wir geben eine Übersetzung in eine CSGS-Signatur an:
\begin{align}
  Agent = \{a\} \\
  supply(a, t) \mapsto B \\
  Baustelle = U \\
  demand(u, t) \mapsto w(u) \\
  budget(x) \mapsto v(x) \\
  kosten(t, n, x, y) \mapsto 0
\end{align}
Dabei wird das KNAPSACK-Problem als ein Spiel mit nur einem Agenten und einem Skilltyp verstanden. Die maximale Kapazität des Rucksacks $B$ ist die Quantität dieses Skilltypes des Agenden. Die Menge der Objekte wird als Menge an Bauaufträgen interpretiert, das Gewicht eines Objektes $w(u)$ als die zur Fertigstellung geforderten Ressourcen und dessen Wert $v(u)$ das Auszahlungsbetrages einer Baustelle bei Fertigstellung.

Intuitiv sorgt die Bedingung \ref{bed1} dafür, dass der Agent seine Ressourcen nicht überschreitet, sowie die Bedingung \ref{bed2}, dass der Agent sein Gewinn maximiert. Beides sind ebenfalls Anforderungen, die von einem Mechanismus erfüllt werden muss, der ein optimales CSG berechnet.
Angenommen es gäbe ein $m\in P$ der aus einem CSGS ein CSG mit einem optimalem Matching berechnet, dann wäre auch $KNAPSACK \in P$. Dieses ist jedoch ein Widerspruch zur Annahme $KNAPSACK \in NP$.
\begin{flushright}
  $\qed$
\end{flushright}

\subsubsection*{Stabilisierung}
\label{stabilisierung}
Wie wir in Abschnitt (\ref{instabil}) gesehen haben, ist die große Koalition instabil. Im Folgenden möchten wir in einem Vorschlag zeigen, wie ein Mechanismus zu Koalitionsbildung funktionieren könnte, der eine stabile große Koalition erzeugt:
Eine Koalition garantiert ihren Koalitionsteilnehmern beim Beitritt einen individuellen Gewinn in Höhe des Shapley Values und verlangt dafür einen hinreichenden Beitrittsdepot $x$. Dieser Kollateralbetrag wird von einer unabhängigen Instanz verwaltet und sichert jeden Agenten $a$ vor dem "Rauswurf" aus der großen Koalition ab, indem diese Instanz $a$ in der gleichen Höhe auszahlt, wie auch ihr Gewinn in der großen Koalition wäre.
Am Schluss wird der Kollateralbetrag wieder anteilig an die Agenten zurück verteilt. Durch wird die große Koalition stabilisiert, da es für Teilkoalitionen nicht mehr lohnt auszusteigen: Der Kollateralbetrag würde um genau den gleichen Wert schrumpfen, wie die Auszahlung der zurückgelassenen Agenten. Der Beitrittsdepot für jeden Agenten für die Große Koalition ist (ohne Beweis)
\begin{equation}
   x=\frac{\sum\limits_{b \in Baustellen} budget(b)} {2*|Agenten|}. 
\end{equation}

\subsubsection*{Mechanismus}
In Abschnitt (\ref{supadd}) sowie (\ref{stabilisierung}) haben wir gezeigt, dass der Beitritt zur großen Koalition die rational richtige Entscheidung für einen Agenten ist, was die Offenlegung aller Skillkapazitäten zur Folge hat.

\subsubsection{Mechanismus}
\label{thmechanism}
In Abschnitt (\ref{supadd}) sowie (\ref{stabilisierung}) haben wir gezeigt, dass der Beitritt zur großen Koalition die rational richtige Entscheidung für jeden Agenten ist, was die Offenlegung aller Agentenressourcen zur folge hat.
% Da wir aus Abschnit (\ref{np}) wissen das das Berechnen eines optimalen Matchings NP hart ist, können wir entweder, wie im praktischen Teil beschrieben, auf ein Approximationsalgorithmus zurückgreifen oder nach der suche des maximums, alle Möglichen Matchings ausprobieren
Die Berechnung eines optimalen Matchings ist entscheidbar, jedoch wie wir aus Abschnitt (\ref{np}) wissen, NP-hart.
Das Berechnen des Shapley-Values ist ebenfalls entscheidbar, jedoch ebenfalls NP-hart\footnote{Für eine Untersuchung der Komplexität der Berechnung des Shapley Values sei auf \cite{conitzer2004computing} verwiesen.}.
So haben wir in diesem Kapitel gezeigt, dass es ein NP-harten Mechanismus gibt, der bei einem gegebenen Setting ein faires und optimales Ergebnis berechnet, welches den Anforderungen der Aufgabenstellung (\ref{task}) gerecht wird.


\subsection{Praktische Ergebnisse}

%%%%%%%%%%%%%%%%%%%%%%%%%%%%%%%%%%%%%%%%
%%%%%%%%%%%%%%%%%%%%%%%% ZUSAMMENFASSUNG
%%%%%%%%%%%%%%%%%%%%%%%%%%%%%%%%%%%%%%%%
\section{Zusammenfassung}
\TODO{Zusammenfassung}


\end{document}
