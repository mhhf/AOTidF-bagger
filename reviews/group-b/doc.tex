% Article template for Mathematics Magazine
% Revised 7/2002  Thanks for Greg St. George
\documentclass[sigconf]{acmart}
\usepackage{amssymb}
\usepackage[ngerman]{babel}
\usepackage[utf8]{inputenc}
\usepackage{amsmath}
\usepackage{amsthm}
\usepackage{graphicx}
\usepackage{tikz}
\usepackage{tikz-cd}
\usepackage{dot2texi}

\usetikzlibrary{shapes,arrows}
\renewcommand{\baselinestretch}{1.2}
\setlength\parindent{0pt}
\setcopyright{none}

\begin{document}

\title{Review Gruppe B \\ von Tobias Behrens}
\author{Denis Erfurt, Tobias Behrens, Abdallah Kadour}

\maketitle

In diesem Review halten wir uns die folgende Struktur: in Abschnitt (\ref{struktur}) wird die Struktur und formale Gestaltung des Papers der Gruppe B untersucht. In Abschnitt (\ref{inhalt}) setzen wir uns auf der inhaltlichen Ebene mit diesem Paper auseinander und enden mit einem zusammenfassenden Fazit in Abschnitt (\ref{fazit}).

\section{Struktur und formale Gestaltung}
\label{struktur}
% Struktur des Textes, Lesbarkeit und Verständlichkeit
Das vorliegende Paper ist sinnvoll strukturiert in die inhaltlichen Hauptteile (i) Hinführung und Einordnung der Problemstellung, (ii) Beschreibung der konkreten Aufgabenstellung, (iii) Bearbeitung der Aufgabenstellung und (iv) Ergebnisse. Diese Strukturierung des Papers wird dem Leser nicht vorgestellt oder begründet. Erst gegen Ende des zweiten Abschnittes wird eine Art Vorgehen beschrieben. Die Formalia des Papers bezüglich Zitation und Formatierung werden ansonsten eingehalten. Der Text ist logisch aufgebaut und dem Inhalt kann der Leser entlang der einzelnen Absätzen gut folgen. Zwischen den Abschnitten werden zum Teil kurze Überleitungen formuliert, ohne dem Leser jedoch eine wirkliche Orientierung im Paper geben. Weiterhin führen die Verknüpfungen zu anderen Abschnitten im Text, die vorher nicht benannt wurden, zu kurzer Irritation beim erstmaligen Lesen.

% Begriffe und Definitionen
Definitionen und Formeln werden im Fließtext eingebettet und nicht davon abgehoben als solche präsentiert und benannt. Dadurch ist es für den Leser schwierig Definitionen wiederzufinden und darauf zu verweisen. Zum Teil sind Definitionen verstreut über die gesamte Arbeit und nicht an einem Ort in der Arbeit organisiert. Dennoch sind sie dem Leser zu erschließen und verständlich. Explizite Definitionen zu Grundbegriffen der Arbeit, zum Beispiel \textit{soziale Wohlfahrt} oder \textit{Nash-Produkt}, fehlen. Algorithmen werden nur in Textform beschrieben und nicht in Pseudocode oder Ähnlichem vorgestellt, sind aber auch so  verständlich. Der Code der implementierten Simulation wird nicht zur Verfügung gestellt, sodass Detailfragen zur Funktionsweise nicht selbstständig vom Leser recherchiert werden können.

% Beispiele, Abbildungen
Die in dem Paper eingefügten Abbildungen visualisieren -- besonders im Abschnitt zu den Ergebnissen -- sinnvoll die vorgestellten Zusammenhänge. Sie werden (bis auf Abbildung 6) alle im Fließtext benannt und gut in den Kontext der Arbeit gestellt. Die Abbildungen sind durchnummeriert und sind mit einem Titel oder zum Teil sogar längeren Erklärung versehen. Ein Beispielszenario, an dem das Auktionsverfahren exemplarisch durchgeführt wird, ist nicht vorhanden, aber für das grundlegende Verständnis auch nicht notwendig.

% Sprache, Stil
Einige Formulierungen sind grammatikalisch nicht korrekt und für den Leser schwer verständlich. Sie fallen auch beim ersten Lesen auf: zum Beispiel ein fehlendes Komma schon im Titel zwischen zwei gleichrangigen Adjektiven. Auch die Verwendung verschiedener Tempi in der anfänglichen Zusammenfassung wirkt beim ersten Lesen unglücklich. Wenige lange Sätze mit mehreren verschachtelten Nebensätzen und Passivkonstruktionen erschweren weiterhin das Lesen und schnelle Erfassen der Inhalte.

Ein Korrekturlesen durch einen unbeteiligten Dritten hätte hier sicherlich Fehler aufgedeckt und Anstöße für bessere Formulierungen geben können. Das Fachvokabular wird zum Teil nicht benutzt: Dem Leser fällt das prominent im drittletzten Absatz auf der ersten Seite auf: Hier werden die \textit{Synergieeffekt} nicht als Superadditivitätseigenschaft identifiziert. Wenige Fachbegriffe werden verwendet, ohne das Verständnis der Autoren von den Begriffen anzugeben (z.B. \textit{Nutzen} in der Einleitung und im Fazit).


\section{inhaltliche Kritik}
\label{inhalt}
% Thema und These
In der anfänglichen Zusammenfassung wird das Thema (Vergleich verschiedener Bietstrategien für ein Auktionsverfahren) klar benannt, motiviert (Optimierung entlang verschiedener Parameter) und in der Einleitung näher begründet (Beschreibung einer optimalen Koalitionsstruktur, die Bietstrategie erzeugen soll).   Durch diese Hinführung in der anfänglichen Zusammenfassung und Einleitung vermitteln die Autoren den Lesern das bearbeitete Problem und regen zum Weiterlesen an. Durch die Einordnung des bearbeiteten Problems in den Kontext zu anderen klassischen Spielen wird eine Anschlussmöglichkeit für den Leser angeboten.

% Forschungsfrage/Evaluation
Die Forschungsfrage wird im weiteren Verlauf der Arbeit bearbeitet, wobei nacheinander einzelne Aspekte begründet herausgearbeitet (Bewertungsstrategien, Coalition Formation, Auktionsverfahren und Bietstrategien) und die theoretischen Ergebnisse dazu ausformuliert werden. Dabei wird immer wieder motiviert, warum diese Untersuchung sinnvoll ist und lassen dem Leser die Entscheidungen für bestimmte Vorgehen nachvollziehen (zum Beispiel in Abschnitt 3.4 des Papers durch ein Beispielszenario). Im vierten Abschnitt der Arbeit werden die Ergebnisse der Evaluation verschiedener Bietstrategien vorgestellt. Anhand der in der Forschungsfrage benannten Parameter werden die Bietstrategien anhand generierter Szenarien untersucht und die Ergebnisse in Grafiken nachvollziehbar dargestellt.

Der Auktionsmechanismus selbst und seine Rolle für die entstandenen Ergebnisse werden nicht weiter untersucht oder bewertet. Im Fließtext werden die konkreten Ergebnisse ausformuliert und im Fazit kritisch reflektiert (Preisdruck kann auch durch Signalisierung nicht für alle Baustellen eliminiert werden) und eine Gewichtung der einzelnen Ergebnisse wird vorgenommen (höhere Gesamtfahrtkosten sind nicht ausschlaggebend). Das hilft dem Leser den Gesamtzusammenhang der Ergebnisse zu verstehen und im Rückblick auf die Forschungsfrage zu bewerten.

% Ausblick
In einem Ausblick zeigen die Autoren offen gebliebene Fragen auf und geben Anregungen für weitere Aktivitäten. Dabei beschränken sie sich vor allem auf die Ausgestaltung der Bietstrategien und zeigt sinnvolle Erweiterungen auf. Jedoch vermissen wir neben der Evaluation der Auktionsform auch Ideen, ob das Auktionsverfahren für weitere Untersuchungen sinnvoll modifiziert oder verändert werden könnte.

% Sonstiges
Der erste Abschnitt des Fazits sticht für mich hervor, da er stringent die vorhergehende Arbeit aufgreift und kurz und klar diskutiert. Als Leser kann man dabei die gelesene Arbeit dabei rekapitulieren und gedanklich zusammensetzen. Genauso überzeugt die Vorstellung von Bewertungsstrategien, die -- auch wenn nicht in Form einer Definition -- mathematisch definiert wird.

Einen negativen Eindruck hinterlässt bei mir die Einleitung, die zwar eine Einordnung in das Forschungsgebiet anbietet, aber vergisst, dass Vorgehen der Autoren und die Strukturierung des Papers vorzustellen. Auch die Bietstrategien, zentraler Untersuchungsgegenstand der Arbeit, hätte mehr Platz durch Erklärung in Textform oder Beispielen eingeräumt werden können, als sie nur in einer Tabelle zu umreißen.

Die Ergebnisse der Arbeit kann auf weitere Domänen übertragen werden und ist nicht spezifisch auf die beschriebene Aufgabenstellung zugeschnitten. Dafür müssten die Ergebnisse in einem Zwischenschritt verallgemeinert werden, im Speziellen die Bewertungsstrategie.

\section{Fazit}
\label{fazit}
Aus dem Bereich \textit{Struktur und formale Gestaltung} (\ref{struktur}) können wir für die Autoren zusammenfassen, dass eine klare und kleinteiligere Organisation des Textes vor dem Schreiben eine bessere Orientierung in der Arbeit -- auch für den Leser -- ergeben hätte. Dafür sollte die Strukturierung der Arbeit auch in der Arbeit vorgestellt, begründet und immer wieder darauf verwiesen werden. Die Vorplanung des Textes kann auch beim Niederschreiben komplexer Zusammenhänge helfen, die in der Arbeit leider an wenigen Stellen durch umständliche Formulierungen schwer verständlich werden. Die Grobstruktur und der Text ist ansonsten gut verständlich und sinnvoll aufgebaut.

Als Leser des Papers können wir für den Bereich \textit{inhaltliche Kritik} (\ref{inhalt}) zusammenfassen, dass durch das schrittweise Motivieren und Bearbeiten der in anfänglichen Zusammenfassung und Einleitung vorgestellten  Problemstellung ein interessantes Paper entstanden ist. Die anschließende Evaluation der theoretischen Vorarbeit reflektiert kritisch die vorherigen Designentscheidungen, konzentriert sich aber nur auf die Bietstrategien und geht nicht auf das Auktionsverfahren an sich ein.

\end{document}
