% Article template for Mathematics Magazine
% Revised 7/2002  Thanks for Greg St. George
\documentclass[sigconf]{acmart}
\usepackage{amssymb}
\usepackage[ngerman]{babel}
\usepackage[utf8]{inputenc}
\usepackage{amsmath}
\usepackage{amsthm}
\usepackage{graphicx}
\usepackage{tikz}
\usepackage{tikz-cd}
\usepackage{dot2texi}

\usetikzlibrary{shapes,arrows}
\renewcommand{\baselinestretch}{1.2}
\setlength\parindent{0pt}
\setcopyright{none}

\begin{document}

\title{Review Gruppe B \\ von Tobias Behrens}
\author{Denis Erfurt, Tobias Behrens, Abdallah Kadour}

\maketitle

In diesem Review halten wir uns die folgende Struktur: in Abschnitt (1) wird die Struktur und formale Gestaltung des Papers der Gruppe B untersucht. In Abschnitt  (2) setzen wir uns auf der inhaltlichen Ebene mit diesem Paper auseinander und enden mit einem zusammenfassenden Fazit in Abschnitt (3).

\section{Struktur und formale Gestaltung}
\label{struktur}
Das vorliegende Paper ist sinnvoll strukturiert in die inhaltlichen Hauptteile (i) Hinführung und Einordnung der Problemstellung, (ii) Beschreibung der konkreten Aufgabenstellung, (iii) Bearbeitung der Aufgabenstellung und (iv) Ergebnisse. Diese Strukturierung des Papers wird dem Leser nicht vorgestellt oder begründet. Die Formalia des Papers bezüglich Zitation und Formatierung werden ansonsten eingehalten.

Der Text ist logisch aufgebaut und dem Inhalt kann der Leser entlang der einzelnen Absätzen gut folgen. Zwischen den Abschnitten werden zum Teil kurze Überleitungen formuliert, ohne dem Leser jedoch eine wirkliche Orientierung im Paper geben. Weiterhin führen die Verknüpfungen zu anderen Abschnitten im Text, die vorher nicht benannt wurden, zu kurzer Irritation beim erstmaligen Lesen.

Definitionen und Formeln werden im Fließtext eingebettet und nicht davon abgehoben als solche präsentiert und benannt. Dadurch ist es für den Leser schwierig Definitionen wiederzufinden und darauf zu verweisen. Zum Teil sind Definitionen verstreut über die gesamte Arbeit und nicht an einem Ort in der Arbeit organisiert. Eindeutige Definitionen zu Grundbegriffen der Arbeit, zum Beispiel \textit{soziale Wohlfahrt}, fehlen.

Die in dem Paper eingefügten Abbildungen visualisieren -- besonders im Abschnitt zu den Ergebnissen -- sinnvoll die vorgestellten Zusammenhänge. Sie werden (bis auf Abbildung 6) alle im Fließtext benannt und gut in den Kontext der Arbeit gestellt. Die Abbildungen sind durchnummeriert und sind mit einem Titel oder zum Teil sogar längeren Erklärung versehen.

Einige Formulierungen sind grammatikalisch nicht korrekt und für den Leser schwer verständlich. Sie fallen auch beim ersten Lesen auf: zum Beispiel ein fehlendes Komma schon im Titel zwischen zwei gleichrangigen Adjektiven. Auch die Verwendung des Futur 2 in der anfänglichen Zusammenfassung wirkt beim ersten Lesen unglücklich. Wenige lange Sätze mit mehreren verschachtelten Nebensätzen und Passivkonstruktionen erschweren weiterhin das Lesen und schnelle Erfassen der Inhalte. Ein Korrekturlesen durch einen unbeteiligten Dritten hätte hier sicherlich Fehler aufgedeckt und Anstöße für bessere Formulierungen geben können.

Das Fachvokabular wird zum Teil nicht benutzt: Dem Leser fällt das prominent im drittletzten Absatz auf der ersten Seite auf: Hier werden die \textit{Synergieeffekt} nicht als Superadditivitätseigenschaft identifiziert. Wenige Fachbegriffe werden verwendet, ohne das Verständnis der Autoren von den Begriffen anzugeben (z.B. \textit{Nutzen} in der Einleitung und im Fazit).


\section{inhaltliche Kritik}



\section{Fazit}
Aus dem Bereich \textit{Struktur und formale Gestaltung} (\ref{struktur}) können wir für die Autoren zusammenfassen, dass eine klare und kleinteiligere Organisation des Textes vor dem Schreiben eine bessere Orientierung in der Arbeit -- auch für den Leser -- ergeben hätte. Dafür sollte sie auch in der Arbeit vorgestellt, begründet und immer wieder darauf verwiesen werden. Die Vorplanung des Textes kann auch beim Niederschreiben komplexer Zusammenhänge helfen, die in der Arbeit leider an wenigen Stellen durch umständliche Formulierungen schwer verständlich werden. Die Grobstruktur und der Text ist ansonsten gut verständlich und sinnvoll aufgebaut.



\end{document}
