\documentclass{beamer}
\usepackage{lmodern}% http://ctan.org/pkg/lm
\usepackage{amssymb}
\usepackage[utf8]{inputenc}
\usepackage{amsmath}
\usepackage{amsthm}
\usepackage{prftree}
\usepackage{graphicx}

\addtobeamertemplate{frametitle}{
   \let\insertframetitle\insertsectionhead}{}
\addtobeamertemplate{frametitle}{
   \let\insertframesubtitle\insertsubsectionhead}{}

\newtheoremstyle{break}
  {\topsep} {\topsep}%
  {}{}%
  {\bfseries}{:}%
  {\newline}{}%
\theoremstyle{break}

\title[] % (optional, only for long titles)
{Coalitional Skill Games}
\author[Denis Erfurt, Tobias Behrens, Abdallah Kadour] % (optional, for multiple authors)
{Denis Erfurt, Tobias Behrens, Abdallah Kadour}

\begin{document}
  \frame{\titlepage}

  \section*{Introduction}
  \subsection*{Aurgabenstellung}

  \begin{frame}[t]{title}
    Aufteilung mit folgenden Eigenschaften:
    \begin{enumerate}
      \item soziale Wohlfahrt maximieren
      \item stabil
      \item fair
    \end{enumerate}

    Forschungshypothese mit Coalition Formation-Bezug
  \end{frame}

  \subsection*{Prämissen}
  \begin{frame}[t]{title}
    \begin{enumerate}
      \item unvollständige Information - Agenten haben keine Information über die Ressourcen anderer Agenten.
      \item rationale agenten arbeiten für ihr eigenes Interesse
    \end{enumerate}
  \end{frame}

  \subsection*{Forschungshypothese}
  \begin{frame}[t]{title}
    Es esistiert ein Mechanismus der für ein beliebiges Coalition Skill Game, eine Stabile und faire zuordnung berechnet, die pareto effizient ist.
  \end{frame}

  \begin{frame}[t]{title}
    \begin{lemma}[Superadditivität von CSG]
      Das CSG ist Superadditiv.
    \end{lemma}
  \end{frame}

  \begin{frame}[t]{title}
    \begin{lemma}[Instabilität]
      Die große Koalition ist instabil.
    \end{lemma}
    Betrachte Baustele B {A:1, B:1} sowie Agenden A1: {A:1}, A2 {B:1}, A3 {B:1}
    Für die Koalition {A1, A2} lohnt es sich aus der großen Koalition auszusteigen, da anteilig ihr gewinn wechst.
  \end{frame}






% etc
\end{document}

