\documentclass[c]{beamer}
\usepackage{lmodern}% http://ctan.org/pkg/lm
\usepackage{amssymb}
\usepackage[utf8]{inputenc}
\usepackage{amsmath}
\usepackage{amsthm}
\usepackage{prftree}
\usepackage{graphicx}

% -- todo ------
\newcommand\todo[1]{\colorbox{blue!15}{\textbf{todo: }#1}\newline}
%\renewcommand\todo[1]{} % deactivate todos

\addtobeamertemplate{frametitle}{
   \let\insertframetitle\insertsectionhead}{}
\addtobeamertemplate{frametitle}{
   \let\insertframesubtitle\insertsubsectionhead}{}

\newtheoremstyle{break}
  {\topsep} {\topsep}%
  {}{}%
  {\bfseries}{:}%
  {\newline}{}%
\theoremstyle{break}

\title[] {AOTidF Bagger Game}
\author[Denis Erfurt, Tobias Behrens] % (optional, for multiple authors)
{Denis Erfurt, Tobias Behrens}

\begin{document}

  \frame{\titlepage} % Folie 1
  
  \section*{Agenda}
  \begin{frame}{title} % Folie 2
    \begin{enumerate}
      \item Einführung
      \begin{itemize}
        \item Aufgabenstellung
        \item Prämissen
        \item Forschungshypothesen
      \end{itemize}
      \item theoretische Ergebnisse
      \begin{itemize}
        \item Superadditivität
        \item NP-Härte
        \item Stabilisierung der großen Koalition
      \end{itemize}
      \item praktische Ergebnisse
      \begin{itemize}
        \item lineares Auktionsverfahren
        \item Ergebnisvergleich zum Shapley Value
      \end{itemize}
    \end{enumerate}
  \end{frame}
  
  
  \section*{Einführung}
  \subsection*{Aufgabenstellung}
  \begin{frame}{title} % Folie 3
    \todo{Aufgabenstellung}
    Aufteilung mit folgenden Eigenschaften:
    \begin{enumerate}
      \item soziale Wohlfahrt maximieren
      \item stabil
      \item fair
    \end{enumerate}
    Forschungshypothese mit Coalition Formation-Bezug
  \end{frame}


  \subsection*{Prämissen} 
  \begin{frame}{title} % Folie 4
    \todo{Prämissen}
    \begin{enumerate}
      \item unvollständige Information - Agenten haben keine Information über die Ressourcen anderer Agenten.
      \item rationale agenten arbeiten für ihr eigenes Interesse
    \end{enumerate}
  \end{frame}
  
  
  \subsection*{Allgemeine Forschungshypothesen}
  \begin{frame}{title} % Folie 5
    \begin{itemize}
    \item Das in der Aufgabenstellung beschriebene Zuordnungsproblem ist superadditiv  und erfordert einen NP-harten Mechanismus.
    \item Die große Koalition als Lösungsstrategie mit Shapley Value als  Auszahlungsvorschrift ist instabil. Wir können eine Erweiterung vorschlagen, um eine stabile große Koalition zu erhalten.
    \item Es existiert ein lineares Auktionsverfahren, das eine Zuordnung untern den Vorgaben approximiert.
    \end{itemize}
  \end{frame}


  \section*{theoretische Ergebnisse}
  \subsection*{Superadditivität}
  \begin{frame}{title} % Folie 6
    \todo{Superadditivität}
    \begin{lemma}[Superadditivität von CSG]
      Das CSG ist Superadditiv.
    \end{lemma}
  \end{frame}
  
  
  \subsection*{NP-Härte}
  \begin{frame}{title} % Folie 8
    \todo{NP-Härte}
    \begin{lemma}[NP-Härte des Problems]
    
    \end{lemma}
  \end{frame}
  
  
  \subsection*{Stabilisierung der großen Koalition}
  \begin{frame}{title} % Folie 9
    \todo{Stabilisierung der großen Koalition}
    \begin{lemma}[Instabilität]

    \end{lemma}
  \end{frame}

  \section*{praktische Ergebnisse}
  \subsection*{lineares Auktionsverfahren}
  \begin{frame}{title} % Folie 9
    \todo{lineares Auktionsverfahren}
  \end{frame}
  
  \subsection*{Ergebnisvergleich zum Shapley Value}
  \begin{frame}{title} % Folie 9
    \todo{Vergleich zum Shapley Value}
  \end{frame}


\end{document}